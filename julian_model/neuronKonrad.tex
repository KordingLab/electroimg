\documentclass{article}

% if you need to pass options to natbib, use, e.g.:
% \PassOptionsToPackage{numbers, compress}{natbib}
% before loading nips_2017
%
% to avoid loading the natbib package, add option nonatbib:
\usepackage[nonatbib,final]{arxiv_nips_2016}

%\usepackage{nips_2017}

% to compile a camera-ready version, add the [final] option, e.g.:
% \usepackage[final]{nips_2017}

\usepackage[utf8]{inputenc} % allow utf-8 input
\usepackage[T1]{fontenc}    % use 8-bit T1 fonts
\usepackage[colorlinks,citecolor=blue,linkcolor=blue]{hyperref}       % hyperlinks
\usepackage{url}            % simple URL typesetting
\usepackage{booktabs}       % professional-quality tables
\usepackage{amsfonts}       % blackboard math symbols
\usepackage{nicefrac}       % compact symbols for 1/2, etc.
\usepackage{microtype}      % microtypography
\usepackage{verbatim}
\usepackage[svgnames]{xcolor}
\usepackage{amsmath,amssymb}
\usepackage{algorithm}
\usepackage{algorithmic}
\usepackage{graphicx}
\usepackage{subcaption}
\usepackage{tikz}
\usetikzlibrary{shapes,arrows,arrows.meta}
\tikzset{%
  >={Latex[width=2mm,length=2mm]},
  % Specifications for style of nodes:
            base/.style = {rectangle, rounded corners, draw=black,
                           minimum width=1.5cm, minimum height=1cm,
                           text centered, font=\sffamily},
  activityStarts/.style = {base, fill=blue!30},
    InExtractPortion/.style = {base, fill=blue!30},
       InfoAggregator/.style = {base, fill=red!30},
    ColGen/.style = {base, fill=green!10},
    Storage/.style = {base, fill=brown!30},
    LPSolve/.style = {base, fill=violet!30},
    ILPSolve/.style = {base, fill=red!30},
    Output/.style = {base, fill=gray!30},
       startstop/.style = {base, fill=red!30},
    activityRuns/.style = {base, fill=green!30},
         process/.style = {base, minimum width=2.5cm, fill=orange!15,
                           font=\ttfamily},
}

\definecolor{Gray}{gray}{0.9}

\pgfdeclarelayer{edgelayer}
\pgfdeclarelayer{nodelayer}
\pgfsetlayers{edgelayer,nodelayer,main}

\tikzstyle{vertex_default}=[circle,fill=White,draw=Black,text width=0.1cm]
\tikzstyle{tree_edge}=[thick,draw=Green]
\tikzstyle{add_edge}=[thick,draw=Red]
\tikzstyle{local_edge}=[thick,draw=Cyan]

\newcommand\ie{\emph{i.e.}}
\newcommand\eg{\emph{e.g.}}
\newcommand\Eg{\emph{E.e.}}
\newcommand\todo[1]{\textcolor{red}{\textbf{#1}}}


\title{Neuron Konrad}

% The \author macro works with any number of authors. There are two
% commands used to separate the names and addresses of multiple
% authors: \And and \AND.
%
% Using \And between authors leaves it to LaTeX to determine where to
% break the lines. Using \AND forces a line break at that point. So,
% if LaTeX puts 3 of 4 authors names on the first line, and the last
% on the second line, try using \AND instead of \And before the third
% author name.

\title{Constructing a Neuron using Column Generation}
\author{
Julian Yarkony\\
Experian Data Lab \\% and Salk Institute$.^0$ \footnotemark[0]\\
San Diego CA \\
\texttt{julian.e.yarkony@gmail.com} 
%(She has NOT yet approved this ICML submission version for distribution with her name on it.  )\\
%DO NOT DISTRIBUTE
}

\begin{document}
% \nipsfinalcopy is no longer used
%
\maketitle
%
\begin{abstract}
We consider the problem of describing neurons in single neuron images.  To do this we map the problem to a problem of constructing a lineage to one %Results for published benchmark sequences demonstrate the superiority of this approach.
%
%We consider the problem of multicut tracking and attack if with various sophisticated optimization algorithms appleid to trucated models with the goal of admitting effienct inference.  
%We apply Benders decomposition in conjunction with column generation in order to tackle the problem of multi-object tracking and instance segmentation concurrently. In this case a track of an object corresponds to a subset of the detections present across time.  We employ the multi-cut formulation of tracking+instance segmentation.  In our case columns correspond to subsets of the detections at a given time.    This lies in contrast to the clear use of column generation where a column describes a complete track including all of its detections across time.  Benders decomposition is used to include inter-termporal interactions.  Our formulation is applied to the multi-cut formulation and can be shown to correspond to a tighter LP relaxation.
%
\end{abstract}
%
\section{Formulation}
\subsection{Tracks }

\begin{itemize}
\item
We describe the set of detections as $\mathcal{D}$ which we index with $d$.  
\item 
We describe the set of tracks as $\mathcal{P}$ which we index with $p$.  
\item 
We use $X \in \{0,1\}^{|\mathcal{D}\times \mathcal{P}|}$ to denote a mapping of detections to tracks where $X_{dp}=1$ indicates that track $d$ is associated with track $p$.  
\item 
A track consists of as sequence of sub-tracks each of $k$ detections where $k$ is a user defined hyper-parameter that trades off model complexity and efficiency of inference. 
\item 
 We use $\bar{X} \in \{0,1\}^{|\mathcal{D}\times \mathcal{P}|}$ to denote a mapping of detections to tracks where $X_{dp}=1$ indicates that track $d$ is associated with track $p$ and detection $d$ is not in the first $k-1$ detections  on the track.  .
 \end{itemize}
 \subsection{Sub-Tracks }
   \begin{itemize}
\item 
We define the set of subtracks as $\mathcal{S}$ which we index with $s$.  A given subtrack has elements $\{s_1,s_2,s_3...s_k\}$ ordered in time from earliest to latest.  
\item
We use $F \in \{ 0,1\}^{|\mathcal{D}| \times |\mathcal{S}|}$ which we index by $d,s$ respectively.   We set $F_{ds}=1$ if and only if detection $d$ is in subtrack $s $.  
\item
We use $F^- \in \{ 0,1\}^{|\mathcal{D}| \times |\mathcal{S}|}$ which we index by $d,s$ respectively.   We set $F_{ds}=1$ if and only if detection $d$ is the final detection on subtrack $s$.  
\item 
We define a mapping of tracks to subtracks using a matrix $S^0 \in \{0,1 \}^{|\mathcal{S}|\times |\mathcal{P}|}$.  We use $S^0_{sp}=1$ to indicate that track $p$ contains subtrack $s$ as neither the start nor the end.  
\item 
In order to describe the first subtrack on a  track we use matrix $S^+ \in \{0,1 \}^{|\mathcal{S}|\times |\mathcal{P}|}$ which we index with $s,p$ where $S^+_{sp}=1$ if and only if subtrack $s$ is the first subtrack on track $p$
\item 
In order to describe the first subtrack on a  track we use matrix $S^- \in \{0,1 \}^{|\mathcal{S}|\times |\mathcal{P}|}$ which we index with  $s,p$ where $S^+_{sp}=1$ if and only if subtrack $s$ is the first subtrack on track $p$

\item 
The subtracks in the sequence that describes a track overlap eachother.  Hence if a sub-tracks $s_1$ is succeeded by another subtrack $s_2$ on a given track then the final $k-1$ elements on $s_1$ are the same as the earliest $k-1$ elements in $s_2$
\item
We use $Q \in \{ 0,1\}^{|\mathcal{S}| \times |\mathcal{S}|}$ which we index by $s_1,s_2$ respectively.   We set $Q_{s_1s_2}=1$ if and only if subtrack $s_1$ can succeed $s_2$.  
\end{itemize}
\subsection{Costs }
We associate tracks with costs with costs usign the folowing noattion.  We use $\Theta \in \mathbb{R}^{\mathcal{P}}$ which we index by $p$ to associate  tracks  with costs.  We use $\Theta_p$ to associate track $p$ with a cost.  
\begin{itemize}
\item We use $\theta \in \mathbb{R}^{|\mathcal{S}|,3}$ which we index by $s/ [+,-,0]$ respectively.  
\item 
We use $\theta_{s+}$ to denote the cost of starting a track at subtrack $s$.  
\item 
We use $\theta_{s-}$ to denote the cost of terminating a track at subtrack $s$. 
\item 
We use $\theta_{s0}$ to denote the cost of including  a subtrack $s$ in a track as neither the start nor the end  . 
 \end{itemize}
 
 We assocaite a track with cost with cost $\Theta_p$ as follows.  
 \begin{align}
 \Theta_p=\sum_{s \in \mathcal{S}}S^+_{sp}\theta_{s+}+S^-_{sp}\theta_{s-}+S^0_{sp}\theta_{s0}
 \end{align}
 
 \subsection{Collection of Tracks}
 
 We describe a collection of tracks that describe a neuron using a vector $\gamma \in \{ 0,1\}^{|\mathcal{P}|}$ which we index with $p$.  We set $\gamma_p=1$ if and only if track $p$ is incuded in the neuron.  
 
 We use $\Gamma$ to describe the set of all possible neurons.  This a subset of $\gamma \in \{ 0,1\}^{|\mathcal{P}|}$ The cost associated with a nueron described by $\gamma$ is defined by the sum of the tracks that compose it.  The selection of the lowest cost neuron is thuse written below   
 
 \begin{align}
 \min_{\gamma \in \Gamma}\sum_p \gamma_p\Theta_p
 \end{align}
 
 \subsection{Feasibility}

We assume that the soma is defined by a subtrack $s_0$ which is a special subtrack that initializes the neuron.  
 
 A track is included or not included :  $\gamma_p \in \{0,1\}$
 
 No two tracks can continue through a given detection
 \begin{align}
 \sum_p \gamma_p\hat{X}_{dp}\leq 1 \quad \forall d
 \end{align}
 
 A detetion can not be part of more than two tracks.  This blocks succession in close proximity.  
  \begin{align}
 \sum_p \gamma_p X_{dp}\leq 2 \quad \forall d
 \end{align}
%
 A track can not split off a subtrack unless that sub-track is already on a track. 
 \begin{align}
 \sum_p \sum_{s}Q_{ss_1}S^+_{s_1p}\gamma_p \leq \sum_p S^0_{sp}\gamma_p %\quad \forall s \neq s_0
 \end{align}
 If a track terminates at a given detection then no detections can start off it.  A strong penalty for ending a track early negates the need for this.  Since this strong penalty has been described  I ignore this constraint in the document.  
 \begin{align}
\sum_p\gamma_p\sum_s(F_{ds}-F^-_{ds}) \leq (1-\sum_p\gamma_p\sum_s S^-_pF^-_{ds})
 \end{align}
 
 \section{LP relaxation}
 
  \begin{align}
 \min_{\substack{\gamma \geq 0\\   \sum_p \gamma_p\hat{X}_{dp}\leq 1\\ \sum_p \gamma_p X_{dp}\leq 2 \\ \sum_p \sum_{s}Q_{ss_1}S^+_{s_1p}\gamma_p \leq \sum_p S^0_{sp}\gamma_p }}\sum_p \gamma_p\Theta_p
 \end{align}
 
 We now take the dual form of this.  We use lagragne multiplers $\lambda^1 \in \mathbb{R}_{0+}^{|\mathcal{D}|}$,$\lambda^2 \in \mathbb{R}_{0+}^{|\mathcal{D}|}$,$\lambda^3 \in \mathbb{R}_{0+}^{|\mathcal{S}|}$ to respreset nteh constraints above in dual form.  
 
 \begin{align}
 \max_{\substack{\lambda^1 \geq 0\\ \lambda^2 \geq 0 \\ \lambda^3 \geq 0 }}-\sum_d(\lambda^1_d+2\lambda^2_d) \\
 \Theta_p +\sum_d \hat{X}_{dp}\lambda^1_d+\sum_d X_{dp}\lambda^2_d -\sum_{s}(S^0_{sp})\lambda^3_s+ \sum_{s_1} \lambda^3_{s_1}\sum_{s_2}Q_{s_1s_2}S^+_{s_2p}\geq 0
 \end{align}
 
 Finding the most violated constraint is a dynamic program.  Many constraints can be generated at once.  
 
 
 \section{augmenting}
 
 To make things easier I susepct adding the following will help.  We will make its multiplier slightly less
 
 \begin{align}
 \sum_{p}\gamma_p(X_{dp}-\hat{X}_{dp})\leq  \sum_{p}\gamma_p \hat{X}_{dp} \forall d \notin s_0
 \end{align}
 This is a weaker constarint that that imposed by  $\lambda^3$ but can be expressed in addition.   It has thebenifit that it operates on a small number of variables $\mathcal{D}$ not $\mathcal{S}$.  We express it with multipliers  $\lambda^4 \in \mathbb{R}^{|\mathcal{D}}$
 
 
 \begin{align}
 \max_{\substack{\lambda^1 \geq 0\\ \lambda^2 \geq 0 \\ \lambda^3 \geq 0 }}-\sum_d(\lambda^1_d+2\lambda^2_d) \\
 \Theta_p +\sum_d \hat{X}_{dp}\lambda^1_d+\sum_d X_{dp}\lambda^2_d -\sum_{s}(S^0_{sp})\lambda^3_s+ \sum_{s_1} \lambda^3_{s_1}\sum_{s_2}Q_{s_1s_2}S^+_{s_2p}+\sum_{d \notin s_0} \lambda^4_d(X_{dp}-2\hat{X}_{dp})\geq 0
 \end{align}
 \section{Dyanmic form}
Finidng the most violated constraint is a dynamic program.
\begin{align}
 \min_p \Theta_p +\sum_d \hat{X}_{dp}\lambda^1_d+\sum_d X_{dp}\lambda^2_d -\sum_{s}(S^0_{sp})\lambda^3_s+ \sum_{s_1} \lambda^3_{s_1}\sum_{s_2}Q_{s_1s_2}S^+_{s_2p}+\sum_{d \notin s_0} \lambda^4_d(X_{dp}-2\hat{X}_{dp})
 \end{align}
 
 We now plug in for $\Theta_p$
 
 \begin{align}
 \min_p \sum_{s \in \mathcal{S}}S^+_{sp}\theta_{s+}+S^-_{sp}\theta_{s-}+S^0_{sp}\theta_{s0} \\
 \nonumber +\sum_d \hat{X}_{dp}\lambda^1_d+\sum_d X_{dp}\lambda^2_d -\sum_{s}(S^0_{sp})\lambda^3_s+ \sum_{s_1} \lambda^3_{s_1}\sum_{s_2}Q_{s_1s_2}S^+_{s_2p}+\sum_{d \notin s_0} \lambda^4_d(X_{dp}-2\hat{X}_{dp})
 \end{align}

The lowest cost track terminating at $s_2$ can be written as follows.  

Let $\ell_{0s}$ be the cost to start and end a track at subtrack $s$.  Let $\ell_{s_1s_2}$ be of the lowest cost track ending ins $s_2$ wiht $s_1$ as its penultimate subtrack.  

\begin{align}
\ell_{0s}=\theta^-_s+\theta^+_s-\lambda^3_s+\sum_{d}F_{ds}(\lambda^4_d+\lambda^2_d)+\sum_{d}F^-_{ds}(\lambda^1_d-2\lambda^2_d)+ \sum_{s_1} \lambda^3_{s_1}Q_{s_1s}
\end{align}

\begin{align}
\ell_{\hat{s}s}=\ell_{\hat{s}}-\theta^-_{\hat{s}}+\theta^0_{\hat{s}}+\theta^-_s-\lambda^3_s+\sum_{d}F_{ds}(\lambda^4_d+\lambda^2_d)+\sum_{d}F^-_{ds}(\lambda^1_d-2\lambda^2_d)+ %\sum_{s_1} \lambda^3_{s_1}Q_{s_1s}
\end{align}
\bibliographystyle{ieee}
\bibliography{bib_inst_track}

\end{document}
